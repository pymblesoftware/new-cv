
%\underline{September 1999 – March 2000} \\							
Company:	 \textbf  {Thompson-CFS}, Dee Why, Sydney, Australia \\
Technologies:	IRIX, DomainOS, Tru64 UNIX, Solaris, WinCenter, VME, VIMIC, MIL-STD-1553 \Cfootnote[blue]{https://en.wikipedia.org/wiki/MIL-STD-1553} , ARINC-429 \Cfootnote[blue]{https://web.archive.org/web/20111029161330/http://www.holtic.com/category/352-arinc-429.aspx}, MIL-STD498, Ada, C, DOORS, Interleaf.  \\
\textbf{		C Programmer. } \\

Comments:	Defense training organization. I developed software to simulate and stimulate the MIL-STD-1553 and AR-Inc 429 buses. The environment was MIL-STD 498 documentation process. The buses interacted with the rest of the environment via VME boards. \\
\\

\noindent\hfil\rule{0.5\textwidth}{.4pt}\hfil
\\


%\underline{June 1999 - July 1999} (One of two concurrent contracts, 
%after hours, part-time, concurrent with SMA below)
      
%Company:	\textbf  {Transport Management Group}, CBD, Sydney, Australia. \\
%Technologies:	Windows/MFC, ORACLE, Windows NT, Visual C++ 6.0, QVCS\\ 
%\textbf{		C++ Programmer. } \\
%Comments:	Train scheduling. I produced graphical reports in MFC/C++, (eg zig-zag graphs that show when trains are scheduled to arrive/depart at points up and down the line).  The environment was ORACLE Pro*C which was wrapped within smart pointers which loaded an internal cache, pre-fetching and lazy-evaluating as required. The development environment was initially extremely chaotic which I resolved to structure. I introduced, set up and maintained, QVCS as no source code control system was used and QVCS (a free/cheap product) was used at SMA where I was working concurrently (together from 6:30 am to 9 pm every day and sleeping through the weekends). \\
%\\
%
%\noindent\hfil\rule{0.5\textwidth}{.4pt}\hfil
%\\

%\underline{November 1998 – September 1999}	(one of two concurrent contracts) \\					
Company:	 \textbf{Scientific Management Associates}, Lane Cove, Sydney, Australia \\
Technologies:	Windows 98, Windows NT, Windows 98, Visual C++ 6.0, DirectX 6.0, QVCS, VME \Cfootnote[blue]{https://en.wikipedia.org/wiki/VMEbus}, MIL-STD-1397 \Cfootnote[blue]{http://www.interfacebus.com/Design\_Connector\_NTDS\_Bus.html}, 3D Studio Max.  \\
\textbf{		C++/3D Game Engine Programmer for Defence Project. } \\

Comments:	This was originally a 6-month contract which was extended to 11 months to coincide with the completion of the project. This was an extremely challenging project which required examining a real piece of equipment (EOSS) a system much like a periscope on a submarine and developing a design to simulate it. A director head with DLTV and Thermal imager sits about three-quarters the way up the mast of the Huon class mine hunters. On the bridge of the ship is a console that was simulated. I had to put 3 video cards into one computer and get all the device drivers to work together. Then I had to get hardware-accelerated Direct-3D to function on two video cards, and load textures and vertexes into each card, while the other video card displayed a menu system, I wrote that mimicked the controls on the real bridge. Some video cards would detect that they were not the primary display device and switch to software rendering. Other video cards would steal vertex lists or not load textures. The thermal imager had “White hot” and “Black Hot” modes and therefore two sets of textures had to be loaded for each object and flipped between them in the scene as the controls were accessed. The daylight TV camera had an intensity control on the touch-sensitive control panel and therefore I had to walk the vertex list on the video card and adjust the lighting intensity of each vertex in the scene. The glow and dim effect was quite spectacular and was quite cool to play with. Additional functionality included socket code to interact with an Instructors station (which someone else wrote). There was also a Digi I/O board added to the machine which enabled digital/analogue conversion of signals.
	The project was completed 2 two weeks ahead of schedule and I spent some time profiling and optimising it as much as possible. The Navy was happy to sign off on the project.

	I tendered a project to replace the MADS (disk packs) on the submarine project. The small tender (roughly \$100,000) was successful but the project (\$100 million plus) was suspended indefinitely (pending a Royal Commission). If the tender was not revoked I would have had additional continuing contracts with SMA.

This was about the same time that I had the BeOS C++ Ray Tracer article published in Doctor Dobbs Journal.  \Cfootnote[blue]{  Russell R., (Nov 1999) "BeRays: A ray tracer for BeOS", Doctor Dobbs Journal. \\   https://drdobbs.com/tools/the-berays-ray-tracer/184411102} \\
\\
\noindent\hfil\rule{0.5\textwidth}{.4pt}\hfil
\\

	
%%\underline{February 1998 – October 1998}.	\\						
%Company:	\textbf {TowerTechnology}, Lane Cove, Sydney, Australia \\
%Technologies:	Solaris, HP-UX, AIX, Digital Unix, Windows NT 4.0/5.0 beta,  
%Windows 98.
%\textbf{		UNIX/Windows NT, SCSI Device driver developer. } \\
%\\
%Comments:	TowerTechnology develops document and image processing systems and workflow solutions. I was responsible for maintaining the device driver code for the medium changers (large mag-optical disk libraries which contain crypts for disks and several drives).	I became an expert on the SCSI bus protocol. I wrote a class factory pattern-based diagnostic tool. The class factory would generate objects of all medium changers that the company supported and dump all kinds of diagnostic information.
%
%	I completed a device driver for Sun Solaris to access the medium changers via a pass-through SCSI device driver, debugged multithreaded kernel panics on the Tru64 platform and debugged faults on the HP-UX, AIX, Solaris, Tru64, and Windows NT drivers. 
%	
%	One of the faults included reworking some of the drivers when the disk capacity increased from 2.6Gb to 5Gb, 32-bit limits were exceeded and block orientated seeks had to be replaced with ioctl()s on the devices.
%	Some faults required some functionality to be moved from the upper layer of the kernel to the lower-level drivers or vice versa
%
%Technologies:	Purify, Clear Case, SunSoft Visual Workshop, GNU C++.
%SCSI-View (SCSI Analyser hardware). RCS.
%Rational Rose, Paradigm Plus, UML. Lotus Notes.
%Windows NT Kernel debugger (i386kd.exe) and crash dumps.
%PA-RISC, PowerPC, Intel and SPARC machine code and assembler. \\
%\\
%
%\begin{wrapfigure}{R}{0.1 \textwidth}
%    \centering
%    \includegraphics[height=0.75in]{cyberswine1.jpg}
%    \caption{CyberSwine game}
%    \label{simulationfigure}
%\end{wrapfigure}
%
%\noindent\hfil\rule{0.5\textwidth}{.4pt}\hfil
%\\
%
%
%
%%\underline{May 1997 - February 1998} \\								
%Company:	\textbf{Brilliant Digital Entertainment},  Double Bay, Sydney, Australia \\ 
%Technologies:	Windows NT 4.0 / Windows95 / Linux,  ISAPI, Windows Registry, MFC, Visual C++ 5, COM, 
%Automation, InstallShield 3/5, IntraBuilder, Borland C++ Builder, 
%Delphi 3, Microsoft SQL server 6.5, Java, JavaScript, Perl, inline 
%80x86 assembler, WinSock32, Plink., SAMBA, PKware. \\
%\textbf{ 		C++ Programmer.  } \\
%
%Comments:   	BDE is a small dynamic games software house. I was involved in various 
%aspects of real-time interactive movies. I took over the installer. I did all of the Unix work and wrote code to interact with other parts of the system including the ticket server. I also did the credit card validation code via Plink. \\
%\\	
%
%\noindent\hfil\rule{0.5\textwidth}{.4pt}\hfil
%\\
%
%
%%\underline{January 1997 - May 1997} 	\\								
%Company:	 \textbf {Scientia Systems},  North Sydney, Australia.  \\
%Technologies:	AIX / SunOS / Solaris / SCO / Windows NT 4.0 / Windows95.  Visual C++ 5.0, Borland C++ Builder, C-ISAM, SAGA-C, ISDN, Win Gate, SMIT, Humming Bird Exceed XDK, Motif, SAMBA, Microsoft TCP/IP, POP3.   \\
%\textbf{ 		C/C++ Programmer.  } \\
%
%Comments: 	Scientia is a software house that produces a scheduler used
%by manufacturing called “Synchro”. The main output is a Gantt chart 
%with the capacity for drag and drop and running under the Motif system. 
%
%I had previously worked for the company in 1988 when it was known as Scientia-Whitehorse. The system was Accounting (invoicing, accounts receivable, payroll) and manufacturing (Just-In-Time (JIT) and MRP-II).\\
%\\
%
%
%\noindent\hfil\rule{0.5\textwidth}{.4pt}\hfil
%\\
%
%
%%\underline{May 1996 - December 1996} \\							
%Company: \textbf{EyeOn Software}, Crows Nest, Sydney, Australia \\
%Technologies:	Windows NT 3.51/4.0 Intel/Alpha. Visual C++ 4.x RISC Visual C++ 5.0 Intel, MFC, MCI, Install Shield, OLE, Windows registry, ISDN, Notes, TCP/IP. \\
%\textbf{		C++ Computer Graphics Programmer  }\\.  
%
%Comments:	The product "Digital Fusion" was an Object Oriented, multi-processor optimised, multi-threaded spline-based, resolution-independent video compositing system.  My role was to design and implement features file format loaders and savers for the majority of graphic file formats (two dozen variants like JPEG, Sun Raster, PNG, TIFF, Gif, etc). I wrote Windows Registry code and various graphic processing code including Sobel and La Placian, edge detection, and blur filters. I wrote MFC/GUI code for custom controls like a “rubbery” range control (like 2 slider controls in the same control, which stretched and contracted at limits), screw control with infinite wrap-around looping behaviour. The entire GUI was based on ray-traced images and was extremely slick. Amiga style “intuition layer” framework so that existing components would benefit from underlying extensions. DLLs could be dropped in so that at load time the system would recognise and register new components. Aspect-orientated/delegate style development. System administration of Lotus Notes, Novell Netware, network, and CISCO router. Build a system with Install Shield. 
%
%For some months all of the staff went to meetings with angel investors, and business partners overseas and attend SIGGRAPH conferences. I was left alone in charge of the company, answering phones, making sales, sourcing suppliers, deciding markups, arranging conferences, transferring money, paying bills, and my wages.\\
%\\
%
%\noindent\hfil\rule{0.5\textwidth}{.4pt}\hfil
%\\
%
%
%%\underline{November 1995 - May 1996}	\\						
%Organisation:	\textbf {Department of Computer Science}. 
%		James Cook University, Townsville, Australia \\
%Technologies:	HTML, CGI, PERL, PASCAL, Novell 3.11, OSF/1, Digital UNIX.
%	ULTRIX, SOLARIS, IRIX, C, PVM, MPI \Cfootnote[blue]{https://www.mpi-forum.org/}. Various supercomputers including SGI, Cray. Processor farms. \\		
%\textbf{		Tutor. Semester-long Contract. Research Assistant/Programmer. Contract. Various supercomputers. } \\
% 
%Comments:	Taught students PASCAL, data structures. Did some WWW development. Wrote a 25-page Literature review on distributed data structures. Wrote a Scalable 3D torus distributed termination simulation on multiple networked workstations representing processing elements or nodes of the torus in PVM. The simulation is an accurate model of a distributed termination algorithm for the Cray T3D massively parallel processor\Cfootnote[blue]{https://en.wikipedia.org/wiki/Cray\_T3D}. Wrote a pseudo device driver for DEC Alpha under Linux.  Modified Linux system to run OSF/1 binaries.
%
%	Dr B. Mans developed a distributed priority queue out of work 
%from his PhD. thesis and work in Scotland. The Message Passing Interface (MPI) library was unavailable for any of the departments’ equipment, 
%so I was asked to rewrite the best part of a large project to use the Parallel Virtual Machine (PVM) library on a mixture of workstation virtual machine groups and supercomputers.
%The project was delivered ahead of schedule and very few modifications were requested. \\
%\\
%
%\noindent\hfil\rule{0.5\textwidth}{.4pt}\hfil
%\\
%
%%\underline{January 1995 - November 1995} \\							
%Company:	 \textbf {Agire}. Townsville, Australia \\
%Technologies:	SCO UNIX/XENIX, SPARC Solaris. \\
%\textbf{		Salesman/Technical support. } \\
%
%Comments:	Having previously worked on small projects in Informix, XENIX and C 
%for AGIRE. I was asked to join as front office sales and handle local technical support while most of the team travelled.  \\
%\\
%
%\noindent\hfil\rule{0.5\textwidth}{.4pt}\hfil
%\\
%
%
%%\underline{January 1994 - December 1994} 	\\					
%Organisation:	\textbf {Department of Psychology}, James Cook University. Townsville, Australia.\\
%Technologies:	DOS, Windows, Turbo PASCAL, Borland C++. \\
%\textbf{		Research Assistant/Programmer. Contract. } \\
%
%Comments:	Stereopsis is the post-processing of  images by the front of the 
%brain giving the 3D effect found by squinting at the images in MAGIC EYE books. The Psychology department was interested in different effects, shapes and reaction times. The initial main focus was to develop different tests for subjects using in-house developed libraries. METAcode for Windows, a real-time multi-pass event logger, that produced graphs of statistics of events, was designed by Dr Ryan and myself. \footnote{Russell, R.,  \& Ryan C (1994) “METAcoder for windows: real-time and multi-pass
%		event logging and analysis in the social and behavioural
%		sciences.” Psychology Teaching Review.} \footnote{Russell, R., \& Ryan C. (1994) “METAcoder for windows: real-time and multi-pass
%		event logging and analysis in the social and behavioural
%		sciences.” Psychology Software News.} \\
%\\
%
%\noindent\hfil\rule{0.5\textwidth}{.4pt}\hfil
%\\
%
%%\underline{January 1990 - December 1993}.\\						
%		Helicopter Pilot, Outback Australia. Consulting as a University student. \\
%		Part-time Real Estate agent.  Various student jobs. \\
%(AGIRE, BTC, FNQEB, MCD Consulting, etc) Short-term contracts.
%Technologies:	Informix, Pick, C-ISAM, Zinc, C, DOS, UNIX, BASIC, 8051, 8052.
%		Windows, ORACLE Pro*C, METAwindows, AS/400, DECSystem10,
%		VMS, CP/M. MP/M, PC-MOS, XENIX
%
%Comments:	While studying and flying I worked on small projects for various companies and organisations. \\
%\\
%
